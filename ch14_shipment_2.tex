
\documentclass[12pt]{beamer}
\usepackage[UTF8]{ctex} % 中文支持
%\usepackage{beamerthemesplit}
\usepackage{appendixnumberbeamer}

%%%%%%%%%%%%%%%%%%添加包
%\usepackage{verbatim}
\usepackage{multicol}
\usetheme{CambridgeUS}
\usecolortheme{lily}  
\usepackage{listings}
\definecolor{myblue}{RGB}{0,0,139}
\definecolor{myred}{RGB}{139,0,0}
\usepackage[utf8]{inputenc}
%\usepackage{tipa} % 用于音标符号
\usepackage{hyperref}
%%%%%%%%%%%%%%%%%%%%%
\usepackage{booktabs}
\usepackage[scale=2]{ccicons}

\usepackage{pgfplots}
\usepgfplotslibrary{dateplot}

\usepackage{xspace}
\newcommand{\themename}{\textbf{\textsc{metropolis}}\xspace}

\title{INTERNATIONAL BUSINESS ENGLISH}
\subtitle{Chapter14 货物运输}
\date{\today}
\author{Chen Huawen}
\institute{Sanming University}
% \titlegraphic{\hfill\includegraphics[height=1.5cm]{logo.pdf}}


\begin{document}

\frame{\titlepage}

\begin{frame}[allowframebreaks]
\frametitle{14.1 预备知识}
\begin{itemize}
\item Packing is of particular importance in foreign trade because goods have to travel long distances before arriving at destination. \\
\small{包装在国际贸易中尤为重要,因为货物需要经过长途运输才能到达目的地。}

\item The seller should ensure that packing effectively protects the goods from damage and sustains a long distance of transportation. \\
\small{卖方应确保包装能有效保护货物免受损坏,并承受长途运输。}

\item Packing can be divided into transport packing (usually known as outer packing) and sales packing (usually known as inner packing). \\
\small{包装可分为运输包装(通常称为外包装)和销售包装(通常称为内包装)。}

\item Sales packing is designed mainly to promote sales which must be attractive to the customers’ and helpful to sales. \\
\small{销售包装的主要目的是促进销售,必须对顾客具有吸引力并有助于销售。}
\end{itemize}
\end{frame}

\begin{frame}[allowframebreaks]
\frametitle{运输与海运}
\begin{itemize}
\item Shipment is an important part in international business. \\
\small{运输是国际贸易的重要组成部分。}

\item It refers to the carriage of the goods from the seller to the buyer, and it is realized by transportation—by sea or by air, by road or by rail. \\
\small{它指将货物从卖方运送至买方,可通过海运、空运、公路或铁路运输实现。}

\item With the expansion of international trade, the container service has become popular. \\
\small{随着国际贸易的扩展,集装箱服务已变得普遍。}

\item As far as foreign trade is concerned, shipment is mostly made by sea. \\
\small{就外贸而言,运输主要通过海运完成。}

\item So this unit mainly introduces the shipment by sea. \\
\small{因此本章主要介绍海运运输。}
\end{itemize}
\end{frame}

\begin{frame}[allowframebreaks]
\frametitle{运输复杂性}
\begin{itemize}
\item Shipment is complicated. \\
\small{运输过程是复杂的。}

\item Negotiation about it involves time of delivery, port of loading and destination, means of transportation, shipping documents, partial shipments and transshipment, name of ship, shipping advice, etc. \\
\small{相关协商涉及交货时间、装运港与目的港、运输方式、运输单据、分批装运与转运、船名、装运通知等。}
\end{itemize}
\end{frame}
\section{什么是信用证?}
\begin{frame}{信用证的定义}
    \begin{itemize}
        \item 信用证(Letter of Credit, L/C)是银行出具的支付保证文件。
        \item 由开证行(Issuing Bank)根据进口商(买方)的要求开具。
        \item 确保出口商(卖方)在满足信用证条款的情况下收到货款。
    \item \textbf{特点}:
    \begin{itemize}
        \item 银行为中间方,承担付款责任。
        \item 以单据为基础,不涉及货物本身。
        \item 是一种自足文件,独立于买卖合同。
    \end{itemize}
\end{frame}

% 2. 信用证的作用
\section{信用证的作用}
\begin{frame}{信用证的作用}
    \begin{itemize}
        \item 对出口商:
        \begin{itemize}
            \item 确保在满足信用证条款的情况下收到货款。
            \item 降低收款风险。
        \end{itemize}
        \item 对进口商:
        \begin{itemize}
            \item 确保出口商按照合同要求发货并提供相关单据。
            \item 提供银行信用保障。
        \end{itemize}
    \end{itemize}
\end{frame}

% 3. 信用证的参与方
\section{信用证的参与方}
\begin{frame}{主要参与方}
    \begin{itemize}
        \item \textbf{开证行(Issuing Bank)}:应进口商要求开立信用证的银行。
        \item \textbf{受益人(Beneficiary)}:通常是出口商,信用证的收款方。
        \item \textbf{通知行(Advising Bank)}:将信用证通知受益人的银行。
        \item \textbf{议付行(Negotiating Bank)}:接受受益人提交的单据并支付货款的银行。
        \item \textbf{付款行(Paying Bank)}:最终支付款项的银行。
    \end{itemize}
\end{frame}

% 4. 信用证的流程
\section{信用证的流程}
\begin{frame}{信用证的流程}
    \begin{enumerate}
        \item 进口商向开证行申请开立信用证。
        \item 开证行开具信用证并发送给通知行。
        \item 通知行将信用证通知出口商。
        \item 出口商按照信用证要求发货并准备单据。
        \item 出口商将单据提交至议付行或付款行。
        \item 银行审核单据无误后支付货款。
        \item 开证行将资金转给议付行或付款行,完成交易。
    \end{enumerate}
\end{frame}

% 5. 信用证的类型
\section{信用证的类型}
\begin{frame}{信用证的类型}
    \begin{itemize}
        \item \textbf{即期信用证(Sight L/C)}:见单即付。
        \item \textbf{远期信用证(Usance L/C)}:在一定期限后付款。
        \item \textbf{可转让信用证(Transferable L/C)}:受益人可以将信用证的部分权益转让给第三方。
        \item \textbf{背对背信用证(Back-to-Back L/C)}:以原信用证为基础,开立新的信用证。
    \end{itemize}
\end{frame}

% 6. 注意事项
\section{注意事项}
\begin{frame}{注意事项}
    \begin{itemize}
        \item 信用证条款必须清晰明确,避免模糊不清的描述。
        \item 出口商需仔细审核信用证,确保单据能够完全符合要求。
        \item 信用证是一种银行信用,但仍需注意开证行的信誉和资金实力。
    \end{itemize}
\end{frame}

% 7. 总结
\section{总结}
\begin{frame}{总结}
    \begin{itemize}
        \item 信用证是国际贸易中重要的支付工具。
        \item 它为买卖双方提供了银行信用保障。
        \item 熟悉信用证的流程和类型是国际贸易成功的关键。
    \end{itemize}
\end{frame}





% 14.1.1
\begin{frame}[allowframebreaks]
    \frametitle{14.1.1 运输相关方}
    \begin{itemize}
    \item There are three parties involved in most movements of goods, the consignor—who sends the goods, the carrier—who carries them and the consignee—who(收货人) receives them at the destination. \\
    在大多数货物运输中涉及三个主要方,即发货人——发送货物的一方,承运人——负责运输的一方,以及收货人——在目的地接收货物的一方。
    \end{itemize}
    \end{frame}
    
    % 14.1.2
    \begin{frame}[allowframebreaks]
    \frametitle{14.1.2 分批装运与转运}
    \begin{itemize}
    \item Generally, partial shipments and transshipment are favorable to the seller, which puts the seller in a better position to perform the relevant contract. \\
    一般而言,分批装运和转运对卖方有利,因为这使卖方在履行相关合同时处于更有利的地位。
    
    \item According to the relevant stipulations of the UCP, transportation documents which appear on their faces to indicate that shipment has been made on the same means of conveyance and for the same journey, if they indicate the same destination, will not be regarded as covering partial shipments. \\
    根据《跟单信用证统一惯例》(UCP)的相关规定,若运输单据表面显示货物由同一运输工具经同一路线运输且目的地相同,则不视为分批装运。
    
    \item If transshipment is necessary in case of no direct or suitable ship available for shipment, clause can be stipulated in sales contract, i.e. partial shipments and transshipment are allowed. \\
    若因无直达船或合适船舶而需转运,可在销售合同中明确条款,例如“允许分批装运和转运”。
    \end{itemize}
    \end{frame}
    
    % 14.1.3
    \begin{frame}[allowframebreaks]
    \frametitle{14.1.3 装运通知}
    \begin{itemize}
    \item The seller should arrange to send shipping advice to the buyer immediately after the shipment finishes and he receives the signed bills of lading from the ship company. \\
    卖方应在完成装运并收到船公司签署的提单后,立即向买方发送装运通知。
    
    \item The shipping advice contains the time of shipment, the ports of loading and destination, means of transportation, partial shipment, transshipment, name of ship, date of effecting shipment and arrival, contract number, etc. \\
    装运通知需包含装运时间、装运港与目的港、运输方式、分批装运、转运、船名、装运及到港日期、合同编号等内容。
    \end{itemize}
    \end{frame}
    
    % 14.1.4
    \begin{frame}[allowframebreaks]
    \frametitle{14.1.4 运输单据}
    \begin{itemize}
    \item Shipping documents refer to Bill of Lading, Insurance Policy, Letter of Credit, and Commercial Invoice, Packing List, along with the export Bill of Exchange. \\
    运输单据包括提单、保险单、信用证、商业发票、装箱单以及出口汇票。
    
    \item Although other documents like Invoice, Certificate of Origin, and Certificates of Inspection will be sent along with the draft, they are not indispensable documents. \\
    尽管其他文件如发票、原产地证书和检验证书会随汇票一同寄送,但它们并非必需单据。
    
    \item Shipping documents only include documents of the ownership, but not all export documents. \\
    运输单据仅包含物权凭证,而非所有出口单据。
    
    \item When negotiating payment of goods, the shipping documents should be given by the seller to the bank. \\
    在议付货款时,卖方需将运输单据提交给银行。
    
    \item And sometimes, when sending the shipping advice to the buyer, the copies or duplicates of these documents should be given, too. \\
    有时,在向买方发送装运通知时,还需附上这些单据的副本或复印件。
    \end{itemize}
    \end{frame}
    
    % 14.1.5
    \begin{frame}[allowframebreaks]
    \frametitle{14.1.5 提单(B/L)}
    \begin{itemize}
    \item Bill of Lading (B/L) is issued by the captain or carrier to testify that the captioned goods have been received or shipped on delivery to a certain place of destination. \\
    提单由船长或承运人签发,用以证明所述货物已接收或已装运至指定目的地。
    \end{itemize}
    \end{frame}

    % B/L 类型部分
\begin{frame}[allowframebreaks]
    \frametitle{提单(B/L)的类型}
    \begin{itemize}
    \item it represents title to the goods, it plays a very important role in international trade. \\
    提单代表货物所有权,在国际贸易中具有重要作用。以下是几种主要提单类型:
    
    \item There are several kinds of B/L: \\
    提单可分为以下类型:
    
    \begin{itemize}
    \item On board B/L (or Shipped B/L) and Received for Shipment B/L \\
    已装船提单(Shipped B/L)与收货待运提单(Received for Shipment B/L)
    
    \item Clean B/L and Unclean (or Foul) B/L \\
    清洁提单(Clean B/L)与不清洁提单(Unclean B/L)
    
    \item Straight B/L, Blank B/L and Order B/L \\
    记名提单(Straight B/L)、不记名提单(Blank B/L)与指示提单(Order B/L)
    
    \item Direct B/L, Transshipment B/L and Through B/L \\
    直达提单(Direct B/L)、转运提单(Transshipment B/L)与联运提单(Through B/L)
    
    \item Original B/L and Copy B/L \\
    正本提单(Original B/L)与副本提单(Copy B/L)
    
    \item Advanced B/L \\
    预借提单(Advanced B/L)
    
    \item Antedated B/L \\
    倒签提单(Antedated B/L)
    
    \item Stale B/L \\
    过期提单(Stale B/L)
    \end{itemize}
    \end{itemize}
    \end{frame}
    
    % 函件写作指南
    \begin{frame}[allowframebreaks]
    \frametitle{函件写作指南与样信}
    \begin{itemize}
    \item The following are the typical writing steps concerning packing: \\
    以下是关于包装函件的典型写作步骤:
    
    \begin{enumerate}
    \item Pleasant opening/ To inform the receiver that you would like to discuss about packing. \\
    友好开头/ 告知收件人需讨论包装事宜。
    
    \item To express the detailed requests such as packing materials and packing cost. \\
    明确具体需求,如包装材料与成本要求。
    
    \item To express the wish that the receiver can accept the package requirements and early reply. \\
    表达希望对方接受包装要求并尽早回复的意愿。
    \end{enumerate}
    
    \item The following are the typical writing steps for the replies concerning packing: \\
    以下是关于包装回复函件的典型写作步骤:
    
    \begin{enumerate}
    \item Express appreciation for receiving letter and your main idea. \\
    对来函表示感谢并阐明己方观点。
    
    \item Write your opinions or solutions to the package. \\
    提出对包装的意见或解决方案。
    
    \item Express the hope to confirm your proposal and pleasant ending. \\
    期望对方确认提案并以友好结尾。
    \end{enumerate}
    
    \item The following are the typical writing steps concerning shipment: \\
    以下是关于装运函件的典型写作步骤:
    
    \begin{enumerate}
    \item Stating that the covering letter of credit has been opened. \\
    声明相关信用证已开立。
    
    \item Explain the necessity and reason for immediate shipment. \\
    说明立即装运的必要性与原因。
    
    \item Advise the impacts of delay in shipment and hope for an early shipment. \\
    告知延迟装运的影响并希望尽早完成装运。
    \end{enumerate}
    
    \item The following are the typical writing steps for the replies concerning shipment: \\
    以下是关于装运回复函件的典型写作步骤:
    
    \begin{enumerate}
    \item Notify the buyer that the goods under the contract have been shipped by/via something on a certain date. \\
    通知买方合同项下货物已于某日通过某方式装运。
    
    \item Inform the buyer what documents have been sent. \\
    告知买方已寄送的单据类型。
    
    \item Hope the goods will arrive or reach in sound condition. \\
    期望货物完好抵达。
    
    \item Appreciate the concluded business and expect to receive further orders. \\
    对达成交易表示感谢并期待后续订单。
    \end{enumerate}
    \end{itemize}
    \end{frame}

    % 14.2.1 买方包装要求
\begin{frame}[allowframebreaks]
    \frametitle{14.2.1 买方对包装的要求}
    \begin{itemize}
    \item \textbf{March 23, 2020} \\
    \textbf{2020年3月23日}
    
    \item \textbf{Dear Phoebe,} \\
    \textbf{尊敬的菲比:}
    
    \item We thank you for your letter of March 22 enclosing Sales Confirmation No. SC2020-102. \\
    感谢您3月22日来函及随附的第SC2020-102号销售确认书。
    
    \item We find that the packing clause in it is not clear enough after going through the contract. \\
    经审阅合同后,我们发现其中包装条款不够明确。
    
    \item In order to avoid future trouble, we would like to make clear in advance our packing requirements as follows: \\
    为避免后续纠纷,我方特提前明确包装要求如下:
    
    \item The item under the contract should be packed in international standard color boxes, 6 color boxes to a corrugated brown carton. \\
    合同项下货物应采用国际标准彩盒包装,每6个彩盒装入一个棕色瓦楞纸箱。
    
    \item On the outer packing please mark our initials BI in a triangle, under which the port of destination and sales confirmation No. SC2020-102 should be stenciled. \\
    外包装上需印刷三角形标志,内含我司缩写BI,其下方需标注目的港及销售确认书号SC2020-102。
    
    \item In addition, marks like FRAGILE, HANDLE WITH CARE, MADE IN CHINA, should also be indicated. \\
    此外,还需标注"易碎品"、"小心搬运"、"中国制造"等标识。
    
    \item We hope that the above requirements could be met and paid special attention. \\
    望贵司满足上述要求并予以特别关注。
    
    \item \textbf{Best regards, Andy Estes} \\
    \textbf{此致敬礼,安迪·埃斯特斯}
    \end{itemize}
    \end{frame}

 
    % 14.2.2 包装要求回复
    \begin{frame}[allowframebreaks]
    \frametitle{14.2.2 对包装要求的回复}
    \begin{itemize}
    \item \textbf{March 24, 2020} \\
    \textbf{2020年3月24日}
    
    \item \textbf{Dear Andy,} \\
    \textbf{尊敬的安迪:}
    
    \item We have received your letter of March 23 concerning the details of packing and shipping marks. \\
    我方已收到贵司3月23日关于包装及运输标志的来函。
    
    \item We will make sure to meet all the requirements. \\
    我方将确保满足所有要求。
    
    \item Packing of 2,400PCS A001 and A003 each under S/C No. SC2020-102 will be packed as you required. \\
    第SC2020-102号销售确认书项下的A001和A003型号各2400件货物,将按贵司要求进行包装。
    
    \item We will stencil the shipping mark on the outer packing as follows: \\
    外包装将印刷如下运输标志:
    \end{itemize}
\end{frame}

\    \begin{frame}
    \begin{verbatim}
    WI\\
    NEW YORK\\
    SC2020-102\\
    C/NO.\\
    \end{verbatim}
\end{frame}

\begin{frame}
    \begin{itemize}
    \item Please let us know whether you are satisfied with the above design of shipping marks. \\
    请确认是否满意上述运输标志设计。
    
    \item Shipment could be made within 30 days after the receipt of your L/C. \\
    收到信用证后30天内可完成装运。
    
    \item \textbf{Best regards,} \\
    \textbf{此致敬礼,}
    
    \item \textbf{Phoebe He} \\
    \textbf{何菲比}
    
    \item \textbf{Sales Manager} \\
    \textbf{销售经理}
    \end{itemize}

    \end{frame}

%%%%%%%14-5

\begin{frame}[allowframebreaks]{14.2.3 催促装运 (Urging Shipment)}
    \textcolor{myblue}{April 25, 2020}
    
    \textcolor{myblue}{Dear Sirs,}
    
    \textcolor{myblue}{Re: S/C No. SC2020-102}
    
    \vspace{0.2cm}
    \textcolor{myblue}{We should like to draw your attention to the captioned Sales Confirmation No. SC2020-102 covering 2,400PCS A001 and A003 each for which we sent to you on April 5 an irrevocable L/C expiration date May 30.}
    
    \textcolor{myred}{兹提请贵方注意第SC2020-102号销售确认书,涵盖A001和A003各2400件。我方已于4月5日向贵方开出不可撤销信用证,有效期至5月30日。}
    
    \vspace{0.2cm}
    \textcolor{myblue}{Up to the present moment no news has come from you about the shipment.}
    
    \textcolor{myred}{迄今为止,我方尚未收到任何关于装运的消息。}
    
    \framebreak
    
    \textcolor{myblue}{As the season is rapidly approaching, our buyers are in urgent need of the goods.}
    
    \textcolor{myred}{由于销售旺季即将来临,我方客户急需此批货物。}
    
    \vspace{0.2cm}
    \textcolor{myblue}{We shall much appreciate it if you effect shipment as soon as possible, thus enabling the goods to arrive here in time to catch the brisk demand at the start of the season.}
    
    \textcolor{myred}{若贵方能尽快安排装运,使货物及时抵达以赶上旺季初期的旺盛需求,我方将不胜感激。}
    
    \vspace{0.2cm}
    \textcolor{myblue}{We trust you will see to it that the order is shipped within the stipulated time, as any delay would cause us no little inconvenience and financial loss.}
    
    \textcolor{myred}{我方相信贵方会确保订单在规定时间内装运,因任何延误都将给我方带来极大不便和经济损失。}
    
    \framebreak
    
    \textcolor{myblue}{Your close cooperation will be highly appreciated.}
    
    \textcolor{myred}{如蒙密切合作,深表感谢。}
    
    \vspace{0.2cm}
    \textcolor{myblue}{Yours faithfully,}
    
    \textcolor{myblue}{Andy Estes}
    
    \textcolor{myred}{此致}
    
    \textcolor{myred}{安迪·埃斯特斯}
    \end{frame}
    \begin{frame}[allowframebreaks]{14.2.3 Urging Shipment: Questions}
        \textcolor{myblue}{\textbf{1. What is the expiration date of the irrevocable L/C mentioned in the letter?}}  
        \begin{itemize}
        \item[A)] April 5, 2020  
        \item[B)] \textcolor{myred}{May 30, 2020}  
        \item[C)] June 30, 2020  
        \item[D)] Not specified  
        \end{itemize}
        
        \textcolor{myblue}{\textbf{Answer: B) May 30, 2020}}  
        \footnotesize\textit{(The letter states: "an irrevocable L/C expiration date May 30.")}
        
        \vspace{0.5cm}
        \textcolor{myblue}{\textbf{2. Why is the buyer urging prompt shipment?}}  
        \begin{itemize}
        \item[A)] To avoid penalties for late payment  
        \item[B)] \textcolor{myred}{To catch the peak demand at the start of the season}  
        \item[C)] To reduce transportation costs  
        \item[D)] To comply with new trade regulations  
        \end{itemize}
        
        \textcolor{myblue}{\textbf{Answer: B) To catch the peak demand at the start of the season}}  
        \footnotesize\textit{(The letter emphasizes "brisk demand at the start of the season.")}
        \end{frame}
        
    % -------------------------
    \begin{frame}[allowframebreaks]{14.2.4 请求分批装运 (Requesting Partial Shipments)}
    \textcolor{myblue}{April 26, 2020}
    
    \textcolor{myblue}{Dear Sir or Madam,}
    
    \vspace{0.2cm}
    \textcolor{myblue}{We feel regretful to inform you that our factory has suffered a flood.}
    
    \textcolor{myred}{很遗憾告知贵方,我方工厂遭遇洪水灾害。}
    
    \vspace{0.2cm}
    \textcolor{myblue}{It is impossible for us to ship 2,400PCS A001 and A003 each ordered by one lot by the end of May, 2020.}
    
    \textcolor{myred}{我方无法在2020年5月底前一次性装运A001和A003各2400件。}
    
    \framebreak
    
    \textcolor{myblue}{We request you to allow us partial shipment, that is 2,000PCS A001 and A003 each within the contracted time and the remaining 400PCS each in June, 2020.}
    
    \textcolor{myred}{恳请贵方允许分批装运:即按合同时间装运A001和A003各2000件,剩余各400件于2020年6月装运。}
    
    \vspace{0.2cm}
    \textcolor{myblue}{We will request you to extend the validity of your L/C to June 30, 2020.}
    
    \textcolor{myred}{我方请求贵方将信用证有效期延长至2020年6月30日。}
    
    \vspace{0.2cm}
    \textcolor{myblue}{Though this is a case of Force Majeure, we are making every effort to recover our production.}
    
    \textcolor{myred}{尽管此次事件属于不可抗力,我方仍将全力恢复生产。}
    
    \framebreak
    
    \textcolor{myblue}{We will appreciate it if you would kindly understand the situation and accept our request.}
    
    \textcolor{myred}{若贵方能理解当前情况并接受我方请求,我方将深表感激。}
    
    \vspace{0.2cm}
    \textcolor{myblue}{Look forward to your early reply.}
    
    \textcolor{myred}{期待贵方早日回复。}
    
    \vspace{0.2cm}
    \textcolor{myblue}{Sincerely yours,}
    
    \textcolor{myblue}{Phoebe He}
    
    \textcolor{myblue}{Sales Manager}
    
    \textcolor{myred}{此致}
    
    \textcolor{myred}{何菲比}
    
    \textcolor{myred}{销售经理}
    \end{frame}

% -------------------------
\begin{frame}[allowframebreaks]{14.2.4 Requesting Partial Shipments: Questions}
    \textcolor{myblue}{\textbf{1. What is the proposed partial shipment plan?}}  
    \begin{itemize}
    \item[A)] Ship 2,400 units in June 2020  
    \item[B)] \textcolor{myred}{Ship 2,000 units by May 2020 and 400 units in June 2020}  
    \item[C)] Cancel the order entirely  
    \item[D)] Delay all shipments until July 2020  
    \end{itemize}
    
    
    
    \vspace{0.5cm}
    \textcolor{myblue}{\textbf{2. What event caused the seller to request partial shipments?}}  
    \begin{itemize}
    \item[A)] A labor strike  
    \item[B)] \textcolor{myred}{A flood at the factory}  
    \item[C)] A sudden increase in demand  
    \item[D)] A shipping company strike  
    \end{itemize}
    
    \vspace{0.5cm}
    \textcolor{myblue}{\textbf{Questions Answer1: B) Ship 2,000 units by May 2020 and 400 units in June 2020}}  
    \footnotesize\textit{(The seller requests "2,000PCS within contracted time and 400PCS in June.")}\\
    
    \vspace{0.5cm}
    \textcolor{myblue}{\textbf{Questions Answer2: B) A flood at the factory}}  
    \footnotesize\textit{(The letter states: "our factory has suffered a flood.")}
    \end{frame}

    %%%%%%%%%%%%%%%%%%14-6

    \section{14.2.5 建议转运}
\begin{frame}[allowframebreaks]
\frametitle{建议转运(中英对照)}
\textbf{Page 1 Content:}

\medskip
\textbf{英文原文:}
Dear Sirs,\\
We have received the L/C No. 6688 you established through Citibank, New York on April 5, covering Contract No. SC2020-102. 

\smallskip
\textbf{中文翻译:}\\
敬启者:\\
我们已收到贵方通过纽约花旗银行于4月5日开立的6688号信用证,涉及合同号SC2020-102。

\medskip
\textbf{英文原文:}
We find that the port of destination should be New York and transshipment is not allowed.

\smallskip
\textbf{中文翻译:}\\
我方注意到目的港应为纽约且不允许转运。

\medskip
\textbf{英文原文:}
However, we are advised by the shipping company that because direct vessels sailing for New York are few and the shipping space has been fully booked up to the end of May.

\smallskip
\textbf{中文翻译:}\\
然而,船公司告知,由于直航纽约的班轮稀少,且舱位已预订至五月底。

\medskip
\textbf{英文原文:}
Under this circumstance, we regret being unable to meet your requirement.

\smallskip
\textbf{中文翻译:}\\
在此情况下,我们遗憾无法满足贵方要求。

\medskip
\textbf{英文原文:}
We suggest you to make transshipment via Los Angeles. In this case, you must bear the additional transshipment charges.

\smallskip
\textbf{中文翻译:}\\
建议贵方通过洛杉矶进行转运。此情形下,贵方需承担额外转运费用。

\framebreak

\textbf{英文原文:}
In order to make sure that the goods will be duly delivered, which is to our mutual benefit, we request that you amend the L/C to allow transshipment.

\smallskip
\textbf{中文翻译:}\\
为确保货物按时交付(此举符合双方共同利益),请修改信用证允许转运。

\medskip
\textbf{英文原文:}
Please take the above into consideration and let us know your decision as soon as possible.

\smallskip
\textbf{中文翻译:}\\
请考虑上述建议并尽快告知决定。

\medskip
\textbf{英文原文:}
Yours sincerely,

\smallskip
\textbf{中文翻译:}\\
此致\\
敬礼
\end{frame}

\section{14.2.6 装运指示}
\begin{frame}[allowframebreaks]
\frametitle{装运指示(中英对照)}
\textbf{英文原文:}
Dear Sirs,\\
Re: Your Sales Confirmation No. SC2020-102 Thank you for informing us that the items under SC2020-102 are now ready for shipment.

\smallskip
\textbf{中文翻译:}\\
敬启者:\\
关于贵方SC2020-102号销售确认书,感谢通知我方该合同项下货物已备妥待运。

\medskip
\textbf{英文原文:}
Since the purchase is made on an FOB basis, you are to load the goods at Shenzhen on the ship appointed by us.

\smallskip
\textbf{中文翻译:}\\
由于本次采购采用FOB条款,请将货物装至我方指定的船舶(深圳港)。

\medskip
\textbf{英文原文:}
Please arrange to send the consignment to Shenzhen for shipment by S.S. Victory due to sail for New York on May 15 from May 10 to 14 inclusive.

\smallskip
\textbf{中文翻译:}\\
请安排将货物发往深圳,以便装运"胜利号"轮(该轮定于5月10日至14日期间启航,5月15日抵达纽约)。

\framebreak

\textbf{英文原文:}
All cartons should be clearly marked and numbered as shown in our official order.

\smallskip
\textbf{中文翻译:}\\
所有纸箱须按我方正式订单所示清晰标注唛头及编号。

\medskip
\textbf{英文原文:}
For further instructions, please contact our forwarding agent, ABC Co., Ltd., New York.

\smallskip
\textbf{中文翻译:}\\
如需进一步指示,请联系我方货运代理——纽约ABC有限公司。

\medskip
\textbf{英文原文:}
We await your shipping advice. Invoices, Packing List and other necessary documents should be sent to us at the same time.

\smallskip
\textbf{中文翻译:}\\
期待贵方装船通知。发票、装箱单及其他必要单据请一并寄送。

\medskip
\textbf{英文原文:}
Yours truly,

\smallskip
\textbf{中文翻译:}\\
此致\\
敬礼
\end{frame}

    \end{document}

